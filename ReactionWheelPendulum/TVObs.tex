\subsubsection*{Change of coordinates to remove $a_1$}
We consider the system 
\[
	\dot{x}_p = A_p(q)x_p + \beta_p(t), \ y=C_px,
\]
where
\[
	\begin{aligned}
		A_p(q) := \begin{bmatrix} 0 & 1 &  0\\ 0 & 0 & -a_1\cos(q)\\ 0 & 0 & 0 \end{bmatrix}, \ C_p := \begin{bmatrix} 1 & 0 & 1 \end{bmatrix}
	\end{aligned}
\]
with the known constant parameter $a_1>0$, measured external signal $q(t)$, and known reference input $\beta_p(t)$. The system is observable for $\cos(q)\ne 0$ and is not observable for $\cos(q)=0$. The goal is to design an observer for $x_p$.

First we define the constant invertible matrix
\[
	T_p :=  \begin{bmatrix} 0 & 0 & -a_1 \\ 0 & 1 & 0\\ 1 & 0 & 1 \end{bmatrix}
\]
and the change of coordinates $x := T_p x_p$. Hence
\[
	\dot{x} = A(q)x + \beta, \ y=Cx,
\]
where
\[
	\begin{aligned}
		A(q) &:= T_p A_p(q) T_p^{-1} = 
		\begin{bmatrix} 0 & 0 & 0\\ \cos(q) & 0 & 0\\ 0 & 1 & 0 \end{bmatrix}, \\
		C &:= C_p T_p^{-1} = \begin{bmatrix} 0 & 0 & 1 \end{bmatrix}, \ \beta:=T_p\beta_p.
	\end{aligned}
\]
When $\cos(q)\ne 0$ the pair $\left(A(q),C\right)$ is observable. However, for $\cos(q)$ the matrix $A$ is in the stair-case observable form, \emph{i.e.} the states $x_{2}$ and $x_{3}$ remain observable while the state $x_{1}$ is not. Note that since the matrix $T_p$ is constant and invertible, the problem of estimation of $x_p$ is equivalent to the problem of estimation of $x$.

\subsubsection*{Switched observer}
Let us rewrite the model as
\[
	\dot{x} = A_\pm x + \beta +B_f f_\pm(x_1,q),
\]
where
\[
	A_{\pm} = \begin{cases}
		A_+ \text{ for } \cos(q)\ge0, \\
		A_- \text{ for } \cos(q)<0,
	\end{cases},
	\ 
	f_{\pm}(x_1,q) = \begin{cases}
		f_+(x_1,q) \text{ for } \cos(q)\ge0, \\
		f_-(x_1,q) \text{ for } \cos(q)<0,
	\end{cases}	
\]
with
\[
	\begin{gathered}
		A_+:=\begin{bmatrix} 0 & 0 & 0 \\ 1 & 0 & 0 \\ 0 & 1 & 0 \end{bmatrix}, A_-:=\begin{bmatrix} 0 & 0 & 0 \\ -1 & 0 & 0 \\ 0 & 1 & 0 \end{bmatrix}, \\
		f_+(x_1,q) = \left(\cos(q)-1\right)x_1, \\ 
		f_-(x_1,q) = \left(\cos(q)+1\right)x_1.
	\end{gathered}
\]
Let $\hat{x}$ be the estimate of $x$. Define $e:=\hat{x}-x$ and
\[
	\Delta f:= f_\pm(\hat{x}_1,q)-f_\pm(x_1,q).
\]
Then for all $q$ we have 
\[
	|\Delta f| = \left(1-|\cos(q)|\right)|e_1|
\]
and 
\[
	|\Delta f|^2 = e_1^2 - \gamma(q)e_1^2, 
\]
where $\gamma(q)=|\cos(q)|\left(2 - |\cos(q)|\right)$. Not that $0\le \gamma(q) \le 1$, where $\gamma(q)=0$ only when $\cos(q)=0$.

We now design our switched observer as
\[
	\dot{\hat{x}} = A_\pm \hat{x} + \beta + B_f f_\pm(\hat{x}_1,q) - L_\pm(\hat{y}-y),
\]
where 
\[
	L_{\pm} = \begin{cases}
		L_+ \text{ for } \cos(q)\ge0, \\
		L_- \text{ for } \cos(q)<0
	\end{cases}
\]
and the values $L_+$ and $L_-$ are to be defined later. The commutation between different dynamics of the observer is governed by $q(t)$, which is a measured signal.

\bigskip

The error dynamics for $\cos(q)\ge0$ is 
\[
	\dot{e} = \left(A_+-L_+C\right)e + B_f \Delta f.
\]
Define $V_+ := e^\top P_+ e$ for some $P_+>0$. Then for $\cos{q}\ge 0$ we have
\[
	\begin{aligned}
		\dot{V}_+ =& e^\top\left(\left(A_+-L_+C\right)^\top P_+ + P_+ \left(A_+-L_+C\right)\right) e 
		+ 2e^\top P_+ B_f \Delta f + \alpha\left(\Delta f\right)^2 - \alpha\left(\Delta f\right)^2 \\
		=& \begin{bmatrix} e^\top & \Delta f \end{bmatrix}^\top
		\underbrace{\begin{bmatrix} \left(A_+-L_+C\right)^\top P_+ + P_+ \left(A_+-L_+C\right) + \alpha I & P_+B_f \\
		B_f^\top P_+ & -\alpha \end{bmatrix}}_{M_+}
		\begin{bmatrix} e \\ \Delta f \end{bmatrix} - \alpha e^\top Q(q) e,
	\end{aligned}
\]
where $\alpha >0$ and
\[
	Q(q) := \begin{bmatrix} \gamma(q) & 0  & 0 \\ 0 & 1 & 0 \\ 0 & 0 & 1 \end{bmatrix} \ge 0.
\]
Let $P_+$, $L_+$ and $\alpha$ be a solution of the LMI $M_+\le 0$. Then we have $\dot{V}_+\le -\alpha e^\top Q(q) e$.

\bigskip

The error dynamics for $\cos(q)<0$ is
\[
	\dot{e} =  \left(A_- - L_-C\right)e + B_f \Delta f.
\]
Define $V_- := e^\top P_- e$ for some $P_->0$. Then for $\cos{q}< 0$ we have
\[
	\begin{aligned}
		\dot{V}_- =& e^\top\left(\left(A_--L_-C\right)^\top P_- + P_- \left(A_--L_-C\right)\right) e 
		+ 2e^\top P_- B_f \Delta f + \alpha\left(\Delta f\right)^2 - \alpha\left(\Delta f\right)^2 \\
		=& \begin{bmatrix} e^\top & \Delta f \end{bmatrix}^\top
		\underbrace{\begin{bmatrix} \left(A_--L_-C\right)^\top P_- + P_- \left(A_--L_-C\right) + \alpha I & P_-B_f \\
		B_f^\top P_- & -\alpha \end{bmatrix}}_{M_-}
		\begin{bmatrix} e \\ \Delta f \end{bmatrix} - \alpha e^\top Q(q) e.
	\end{aligned}
\]
Define the unitary matrix
\[
	R:=\begin{bmatrix} -1 & 0 & 0 \\  0 & 1 & 0 \\ 0 & 0 & 1 \end{bmatrix}.
\]
It holds $A_- = RA_+R$, $CR=C$, and $RB_f=B_f$. Then it follows that the choice
\[
	P_- = RP_+R>0, \ L_- = RL_+,
\]
yields
\[
	M_- = \operatorname{diag}(R,1)M_+\operatorname{diag}(R,1) \le 0.
\]
Thus it holds $\dot{V}_-\le - \alpha e^\top Q(q) e$. Therefore, in both cases, the switched dynamics is stable for each fixed value of the commutation signal. Let us analyze what happens under switching.

\bigskip

To this end assume now that the trajectory $q(t)$ crosses the border $\cos(q)=0$ in isolated points only. Define the Lyapunov function 
\[
	V(e) = \begin{cases}
		V_+(e) \text{ for } \cos(q) \ge 0, \\
		V_-(e) \text{ for } \cos(q) < 0.
	\end{cases}
\]
Let $\lambda_{m}$ and $\lambda_{M}$ be the minimum and the maximum eigenvalues of the matrix $P_+$, respectively, and note that $P_-$ has the same eigenvalues. It holds
\[
	\lambda_{m} |e|^2 \le V(e) \le \lambda_{M} |e|^2.
\]
and
\[
	\dot{V} \le -\alpha e^\top Q(q) e \le -\alpha \gamma(q)e^\top e \le -\frac{\alpha \gamma(q)}{\lambda_{M}}V = -\eta(q)V.
\]
Let $\lambda_{\Delta}\ge0$ be the  maximum absolute value of the eigenvalues of the matrix $P_+ - RP_+R$. Then we have that during the commutation the variation of the Lyapunov function admits the following upper estimate:
\[
	|\Delta V| = |V_+ - V_-| = |e^\top \left(P_+-RP_+R\right)e| \le \lambda_{\Delta} e^\top e \le \mu V,
\]
where
\[
	\mu:= \frac{\lambda_{\Delta}}{\lambda_{m}}.
\]

To proceed we have to impose an assumption on the trajectory $q(t)$: the trajectory should not cross singular points, where $\cos(q(t))=0$, too often and should not remain in a small vicinity of these points (having $\gamma(q)$ small) for long time. More formally, this assumption can be formulated as follows.

\noindent\emph{Assumption}.

For the trajectory $q(t)$ there exist $T_q>0$ and $\kappa>0$ such that for all $t_0$ it holds:
\begin{itemize}
	\item during the time interval $[t_0, \; t_0+T_q]$ the trajectory $q(t)$ crosses the border $\cos(q)=0$ not more then $n_q$ times;
	\item during the time interval $[t_0, \; t_0+T_q]$ the trajectory $q(t)$ satisfies
	\[
		\int_{t_0}^{t_0+T_q}\eta(q(\tau))d\tau \ge n_q\ln\left(1+\mu\right) + \kappa T_q.
	\]
\end{itemize}

If the Assumption holds, then for any $t_0$ we have
\[
	V(t_0+T_q) \le V(t_0) \left(1+\mu\right)^{n_q} e^{-\int_{t_0}^{t_0+T_q}\eta(q(\tau))d\tau} \le e^{-\kappa T_q} V(t_0) <V(t_0)
\]
and for all $t\in [t_0, \; t_0+T_q]$
\[
	V(t) \le V(t_0)\left(1+\mu\right)^{n_q}.
\]
Then $V(t) \to 0$, and $\kappa$ may be considered as the exponential rate of convergence.

\noindent\emph{Remark}. The Assumption can be relaxed by choosing $T_q$ and $\kappa$ non-uniformly in $t_0$ yielding asymptotic convergence instead of exponential. 

Things TODO:
\begin{itemize}
\item Feasibility analyzes for the LMI;
\item More general problem statement  
\end{itemize}
